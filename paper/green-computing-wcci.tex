\documentclass[a4paper]{article}
\usepackage{hyperref}
\begin{document}

\title{Engineering greener software}

\author{JJ Merelo-Guervós\\
  Department of Computer Engineering, Automatics and Robotics\\
  University of Granada (Spain)\\
  Email: {\sf jmerelo@ugr.es} }


\maketitle

\section{Conference}

Hybrid. Most examples will be for CEC, but the principles are valid for all conferences.

\section{Abstract}

{\em Green} computing is a general term that describes a host of techniques that
try to minimize the carbon footprint of software applications. As such, it is
not a single body of knowledge, but a series of best practices that help reduce
energy consumption relying on the features of any of the different layers that
are exercised by software applications. This represents a challenge at the time
of designing a comprehensive syllabus that would help students develop the
series of skills needed to identify energy bottlenecks and eliminate them. In
this poster we will describe the different concepts involved, and how they will
be delivered to guarantee the achievement of learning objectives.  Green
computing deals with reducing the environmental impact of the creation and use
of computing resources. From the software perspective, it proposes maximizing
the amount of work done for every unit of energy spent. But to achieve that, how
energy is spent across all the different computing layers need to be assessed
and understood.

The challenge of introducing energy efficiency concerns into software
engineering has been recently acknowledged by the joint IEEE/ACM task
force. Environmental concerns is one of the skills mentioned in the draft
competencies in software engineering as part of the needed ``behavioral
attributes'' as well as in the Master's degree in Information Systems (IS), as
part of the IS Strategy and Governance competence; the computer engineering set
of draft competencies includes it as part of the Systems Resource Management
subject matter.

\section{Learning objectives}

The final objective would be to get the attendant to understand how choices of
hardware, software platform and algorithm parameters will affect the
environmental impact of the workload that is going to be created or refactored
and which best practices need to be involved to reduce that impact, including,
in many cases, how to deal with the energy/precision tradeoff. Underneath, a
practical understanding of the energy management in modern computer systems is
also an objective. Finally, the challenges of energy profiling algorithm
implementations will be also presented.

\section{Outline}

\begin{itemize}
\item {\bf Understanding the hardware}: Computing units (CPU, GPU, memory) and
  their energy profiles. Energy-wise heterogeneous architectures. Energy
  consumption sensors and standard APIs to get measurements from them
  (e.g. RAPL). Interfaces to the computing power configuration and
  administration system (e.g., ACPI). ({\em 20 minutes})
\item {\bf Understanding how the workload spends energy}: software profiling,
  methodologies, and tools for energy profiling (PowerMeter, {\sf pinpoint},
  {\sf perf}, hardware meters). This will help the student identify bottlenecks
  from the point of view of performance as well as energy consumption; also find
  out how this consumption scales with workload size. ({\em 20 minutes})
\item {\bf Energy profiling an algorithm implementation}, on how to find out how
  much energy an algorithm is spending. ({\em 20 minutes})
\item {\bf Refactor for reduction of energy footprint}: once the bottlenecks
  have been identified and specific benchmarks developed to measure the energy
  footprint, the student needs to work across the board to reduce the footprint:
  choosing the computing platform where possible, configuring the application to
  work on specific computing units, choosing the programming language toolchain
  that reduces energy consumption (such as the compiler or interpreter) or
  configuring it, change in data structures used to store and process data or
  leveraging of multi-threading or symmetric multiprocessing capabilities. ({\em
    30 minutes})
\end{itemize}

\section{Expected audience size}

Around 25-30 persons.

\section{Biographical sketch}

JJ Merelo obtained a degree in theoretical physics at the University of Granada,
where he obtained a PhD in 1994 and is currently a full professor at the
department of Computer Engineering, Automatics and Robotics. Besides being a
frequent open-source contributor, he has been working as a senior software
engineer. For the last 3-4 years, he has been PI or partipant in grants related
to energy optimization, focused on evolutionary algorithms, and has published
several papers on the subject, as well as a tutorial with the same titiel in
SAC'25 Catania (Italy).


\end{document}
\endinput

