\documentclass[a4paper]{article}
\usepackage{hyperref}
\begin{document}

\title{Engineering greener software}

\author{JJ Merelo-Guervós\\
  Department of Computer Engineering, Automatics and Robotics\\
  University of Granada (Spain)\\
  Email: {\sf jmerelo@ugr.es} }


\maketitle

\section{Abstract}

{\em Green} computing is a general term that describes a host of techniques that
try to minimize the carbon footprint of software applications. As such, it is
not a single body of knowledge, but a series of best practices that help reduce
energy consumption relying on the features of any of the different layers that
are exercised by software applications. This represents a challenge at the time
of designing a comprehensive syllabus that would help students develop the
series of skills needed to identify energy bottlenecks and eliminate them. In
this poster we will describe the different concepts involved, and how they will
be delivered to guarantee the achievement of learning objectives.  Green
computing deals with reducing the environmental impact of the creation and use
of computing resources. From the software perspective, it proposes maximizing
the amount of work done for every unit of energy spent. But to achieve that, how
energy is spent across all the different computing layers need to be assessed
and understood.

The challenge of introducing energy efficiency concerns into software
engineering has been recently acknowledged by the joint IEEE/ACM task
force. Environmental concerns is one of the skills mentioned in the draft
competencies in software engineering as part of the needed ``behavioral
attributes'' as well as in the Master's degree in Information Systems (IS), as
part of the IS Strategy and Governance competence; the computer engineering set
of draft competencies includes it as part of the Systems Resource Management
subject matter.

And this engineering starts with understanding how energy management works in
current hardware, and how this interferes with precise measurement of how much
energy has been spent by a particular workload.

The last part of the tutorial will be devoted to understanding how different
evolutionary algorithms spend energy, and how energy performance can be taken
into account in meta-algorithms such as hyperheuristics.

\section{Tutorial description}

We will first focus on the kind of learning objectives we have in mind for this
tutorial, to then outline how these objectives are going to be progressively
presented.

\subsection{Learning objectives}

The final objective would be to get the attendant to understand how choices of
hardware, software platform and algorithm parameters will affect the
environmental impact of the evolutionary algorithm workload that is going to be
created or refactored and which best practices need to be involved to reduce
that impact, including, in many cases, how to deal with the energy/precision
tradeoff. Underneath, a practical understanding of the energy management in
modern computer systems is also an objective. The challenges of energy
profiling algorithm implementations will be also presented; finally, how the
{\em noisy} nature of energy measurements needs to be taken into account for
either online or offline optimization of the algorithms.

\subsection{Outline}

\begin{itemize}
\item {\bf Understanding the hardware}: Computing units (CPU, GPU, memory) and
  their energy profiles. Energy-wise heterogeneous architectures. Energy
  consumption sensors and standard APIs to get measurements from them
  (e.g. RAPL). Interfaces to the computing power configuration and
  administration system (e.g., ACPI). ({\em 15 minutes})
\item {\bf Understanding how the workload spends energy}: software profiling,
  methodologies, and tools for energy profiling (PowerMeter, {\sf pinpoint},
  {\sf perf}, hardware meters). This will help the student identify bottlenecks
  from the point of view of performance as well as energy consumption; also find
  out how this consumption scales with workload size. ({\em 15 minutes})
\item {\bf Methodologies for energy and power measurement}, that is, how
  experiments need to be designed in order to have as precise a measurement of
  energy expenses as possible.
\item {\bf Energy profiling an algorithm implementation}, on how to find out how
  much energy an algorithm is spending. ({\em 20 minutes})
\item {\bf Refactor for reduction of energy footprint}: once the bottlenecks
  have been identified and specific benchmarks developed to measure the energy
  footprint, the student needs to work across the board to reduce the footprint:
  choosing the computing platform where possible, configuring the application to
  work on specific computing units, choosing the programming language toolchain
  that reduces energy consumption (such as the compiler or interpreter) or
  configuring it, change in data structures used to store and process data or
  leveraging of multi-threading or symmetric multiprocessing capabilities. ({\em
    30 minutes})
\end{itemize}

\subsection{Tutorial categorization}

This is an introductory tutorial for people with basic knowledge of statistical
techniques as well as programming; I would say that covers 99\% of the congress
attendants.

\section{Potential target audience}

That is difficult to say. There are many researchers investigating on the
subject, and it could be potentially interesting for anyone in the area of
hyperheuristics. It is also targeted at people implementing evolutionary
algorithms, which is basically everyone.

\section{Expected number of participants}

Other tutorials I have given have gathered 25-30 persons. I guess that
potentially all students and postdocs could be interested, as well as
established hyperheuristic researchers.

\section{Previously held versions on the tutorial}

A tutorial with the same title in was delivered in ACM SAC'25 Catania (Italy),
it included the state of the art as known back then. We have published since
then several papers, and our experimental techniques as well as methdologies are
much more advanced right now.

I also coordinated a summer school with the same topic, in Spain in the summer
of 2024; my (introductory) lesson was in the same general area.

\end{document}
\endinput

