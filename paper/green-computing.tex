\documentclass[a4paper]{article}
\usepackage{hyperref}
\begin{document}

\title{Engineering greener software}

\author{JJ Merelo-Guervós}


\maketitle

\section{Duration}

Proposed for a half-day (2 hours) tutorial.

\section{Abstract}

{\em Green} computing is a general term that describes a host of techniques that
try to minimize the carbon footprint of software applications. As such, it is
not a single body of knowledge, but a series of best practices that help reduce
energy consumption relying on the features of any of the different layers that
are exercised by software applications. This represents a challenge at the
time of designing a comprehensive syllabus that would help students develop the
series of skills needed to identify energy bottlenecks and eliminate them. In
this poster we will describe the different concepts involved, and how they will
be delivered to guarantee the achievement of learning objectives.
Green computing \cite{kurp2008green} deals, in general, with reducing the
environmental impact of the creation and use of computing resources. From the
software perspective, it proposes maximizing the amount of work done for every
unit of energy spent. But to achieve that, how energy is spent across
all the different computing layers need to be assessed and understood.

This is why getting the student to achieve a certain amount of understanding of
the different process involved, methodologies needed to carry out that
assessment, and eventually design your code from the ground up or refactoring it
to make it {\em greener} is a challenge.

\section{Motivation, target audience and interest for the SAC community}

The challenge of introducing energy efficiency concerns into software
engineering has been recently acknowledged by the joint IEEE/ACM task force in
\cite{cc2020}. Environmental concerns is one of the skills mentioned in the
draft competencies in software engineering as part of the needed ``behavioral
attributes'' as well as in the Master's degree in Information Systems (IS), as
part of the IS Strategy and Governance competence; the computer engineering set
of draft competencies includes it as part of the Systems Resource Management
subject matter. This again shows the inter-disciplinary content that needs to be
considered even if the focus is on software engineering.

Target audience includes, but is not limited to, \begin{itemize}
  \item Computer scientists that want to create more sustainable implementations
    for their applications and algorithms.
  \item Software engineers with some experience that want to be familiarized
    with best practices in energy efficiency.
    \item Tertiary education teachers that want to introduce energy consumption
      reduction into courses in software engineering.
\end{itemize} 

As indicated above, the joint IEEE/ACM task force has indicated their interest
in introducing this kind of content into software engineering students.

\section{Outline}

 The syllabus proposed
would, then, would be as follows:\begin{itemize}
\item {\bf Understanding the hardware}: Computing units (CPU, GPU, memory) and
  their energy profiles. Energy-wise heterogeneous architectures. Energy
  consumption sensors and standard APIs to get measurements from them
  (e.g. RAPL). Interfaces to the computing power configuration and
  administration system (e.g., ACPI). This will help the student to understand
  how the workload exercises various parts of the hardware, and why it does
  so. ({\em 30 minutes})
\item {\bf Understanding how the workload spends energy}: software profiling,
  methodologies, and tools for energy profiling (PowerMeter, {\sf pinpoint},
  {\sf perf}, hardware meters). This will help the student identify bottlenecks
  from the point of view of performance as well as energy consumption; also find
  out how this consumption scales with workload size. ({\em 1 hour})
\item {\bf Refactor for reduction of energy footprint}: once the bottlenecks
  have been identified and specific benchmarks developed to measure the energy
  footprint, the student needs to work across the board to reduce the footprint:
  choosing the computing platform where possible, configuring the application to
  work on specific computing units, choosing the programming language toolchain
  that reduces energy consumption (such as the compiler or interpreter) or
  configuring it, change
  in data structures used to store and process data or leveraging of
  multi-threading or symmetric multiprocessing capabilities. ({\em 30 minutes})
\end{itemize}


\section{Specific goal and objectives}

The final objective would be to get the attendant to understand how choices of
hardware and software platform from the ground up will affect the environmental
impact of the workload that is going to be created or refactored and which best
practices need to be involved to reduce that impact. Underneath, a practical
understanding of the energy management in modern computer systems is also an
objective.

\section{Expected background of the audience}

They need to have knowledge in software development, and at least some knowledge
of the main parts of a computer and what their function is.

\section{Biographical sketch}

JJ Merelo obtained a degree in theoretical physics at the University of Granada,
where he obtained a PhD in 1994 and is currently a full professor at the
department of Computer Engineering, Automatics and Robotics. Besides being a
frequent open-source contributor, he has been working as a senior software
engineer. His GitHub profile is \url{https://github.com/JJ}. His Google Scholar
profile https://scholar.google.com/citations?user=gFxqc64AAAAJ shows several
hundred publications, and close to 10000 citations.

For the last 3-4 years, he has been PI of grants related to energy optimization,
focused on evolutionary algorithms, and has published several papers on the
subject \cite{DBLP:conf/icsoft/GuervosGC23}. He has also coordinated a summer
school on green software development in summer 2024, within the framework of the
summer schools of the University of Granada.

He has ample experience in coordinating summer schools, such as the two
SigEVO summer schools that happened in Osaka and later in Prague. As a teacher,
he has got 36 years experience, including tutorials and presentations in
conferences such as GECCO, PPSN or CEC/WCCI.

\section{Audiovisual equipment}

Overhead projector.

\section{Teaching material}

\begin{itemize}
  \item Teaching how to engineer greener software:
    \url{https://digibug.ugr.es/handle/10481/90506}
  \item Material for the summer course (in Spanish)
    \url{https://jj.github.io/cemed-green-software/} including his own
    presentation
    \url{https://jj.github.io/cemed-green-software/preso/\%C3\%A1gil.html}
\end{itemize}

\section{Tutorial history}

    This is the first time it has been proposed as a tutorial.

\bibliographystyle{plain}
\bibliography{energy.bib,dev.bib}

\end{document}
\endinput

